\documentclass[a4paper,10pt,bibtotoc,abstracton,oneside,noindent,DIV15]{scrartcl}
\usepackage{listings,url}
\usepackage{pslatex,color}
\usepackage[OT1]{fontenc}

\definecolor{lightlightgray}{gray}{0.9}
\definecolor{OliveGreen}{cmyk}{0.64,0,0.95,0.40}

%\usepackage{tex4ht}
\newcommand{\Css}[1]{}

\begin{document}

\lstset{
language=C,                             % Code langugage
basicstyle=\ttfamily,                   % Code font, Examples: \footnotesize, \ttfamily
keywordstyle=\color{OliveGreen},        % Keywords font ('*' = uppercase)
commentstyle=\color{gray},              % Comments font
%numbers=left,                           % Line nums position
%numberstyle=\tiny,                      % Line-numbers fonts
%stepnumber=1,                           % Step between two line-numbers
%numbersep=5pt,                          % How far are line-numbers from code
backgroundcolor=\color{lightlightgray}, % Choose background color
frame=none,                             % A frame around the code
tabsize=2,                              % Default tab size
captionpos=b,                           % Caption-position = bottom
breaklines=true,                        % Automatic line breaking?
breakatwhitespace=false,                % Automatic breaks only at whitespace?
showspaces=false,                       % Dont make spaces visible
showtabs=false,                         % Dont make tabls visible
columns=flexible,                       % Column format
morekeywords={__global__, __device__},  % CUDA specific keywords
}

\lstset{breaklines=true}

\Css{div.lstlisting{font-family: monospace;
    white-space: nowrap; margin-top:0.5em;
    margin-bottom:0.5em;
    color: blue;}}

\Css{ol li { padding: 0; margin: 0.5em; }}

%\Configure{enumerate}{}\Css{li ol{margin-top:5em; margin-bottom:5em;}}

%\lstset{language=Java,captionpos=b,tabsize=3,frame=lines,keywordstyle=\color{blue},commentstyle=\color{darkgreen},stringstyle=\color{red},numbers=left,numberstyle=\tiny,numbersep=5pt,breaklines=true,showstringspaces=false,basicstyle=\footnotesize,emph={label}}

\section{Setup instructions}

\subsection{Dependencies}
Install Hadoop (tested with Hadoop 0.20.2) and Pig. Note that some of the
example scripts require the latest release of Pig (currently 0.10.0) to
be installed.

\subsection{Environment variables}
\begin{enumerate}
\item Set {\tt HADOOP\_HOME} and {\tt PIG\_HOME} to the installation
directories of Hadoop and Pig, respectively, and {\tt SEQPIG\_HOME} to
the installation directory of SeqPig. On a Cloudera Hadoop installation
with a local installation of the most recent Pig release, this would be
done for example by
\begin{lstlisting} 
export HADOOP_HOME=/usr/lib/hadoop
export PIG_HOME=/root/pig-0.10.0
export SEQPIG_HOME=/root/seqpig 
\end{lstlisting}
%
\item To make life simpler also add the shell script directory to your path PATH:
%
\begin{lstlisting} 
export PATH=${PATH}:${SEQPIG_HOME}/bin
\end{lstlisting}
%
\item Finally, for convenience add the following line to your {\tt .bashrc}:
%
\begin{lstlisting} 
alias pig='${PIG_HOME}/bin/pig -Dpig.additional.jars=${SEQPIG_HOME}/lib/hadoop-bam-5.0.jar:${SEQPIG_HOME}/build/jar/SeqPig.jar:${SEQPIG_HOME}/lib/seal.jar:${SEQPIG_HOME}/lib/picard-1.76.jar:${SEQPIG_HOME}/lib/sam-1.76.jar -Dudf.import.list=fi.aalto.seqpig' 
\end{lstlisting}
%
\end{enumerate}

\subsection{Instructions for building SeqPig.jar}

\begin{enumerate}
\item Download hadoop-bam-5.0 from \url{https://sourceforge.net/projects/hadoop-bam/}.

\item Download and compile the latest biodoop/seal git master version from
 \url{http://biodoop-seal.sourceforge.net/}. Note that this requires
 setting {\tt HADOOP\_BAM} to the installation directory of hadoop-bam.

\item Inside the cloned git repository ({\tt \$SEQPIG\_HOME}), create a
{\tt lib/} subdirectory and copy the following jar files contained in the
hadoop-bam release to this location:
%
\begin{lstlisting} 
    seal.jar     hadoop-bam-5.0.jar     sam-1.76.jar     picard-1.76.jar
\end{lstlisting}
%
Note: the Picard and Sam jar files are contained in the hadoop-bam release
for convenience.

\item Run {\tt ant} to build {\tt SeqPig.jar}.
%\begin{enumerate}
%\item test
%\item test2
%\end{enumerate}
\end{enumerate}

\subsection{Usage}

\subsubsection{Pig grunt shell for interactive operations}
Assuming that all the environment variables have been set correctly, it suffices
to start the grunt shell via
%
\begin{lstlisting}
pig
\end{lstlisting}
%
\subsubsection{Starting scripts from the command line for non-interactive use}
Alternatively to using the Pig grunt shell (which can lead to delays due to
Hadoop queuing and execution delays), users can write scripts that are
then submitted to Pig/Hadoop for execution. This type of execution has
the advantage of being able to handle parameters, for example for input
and output files. See /scripts inside the seqpig directory and the
examples ind the doc/ directory.
\end{document}