
\section{Installation}

\subsection{Dependencies}
Install Hadoop (tested with Hadoop 0.20.2) and Pig. Note that some of the
example scripts require the latest release of Pig (currently 0.10.0) to
be installed.

\subsection{Environment variables}
\label{sect:install_env}
\begin{enumerate}
\item Set {\tt HADOOP\_HOME} and {\tt PIG\_HOME} to the installation
directories of Hadoop and Pig, respectively, and {\tt SEQPIG\_HOME} to
the installation directory of SeqPig. On a Cloudera Hadoop installation
with a local installation of the most recent Pig release, this would be
done for example by
\begin{lstlisting} 
export HADOOP_HOME=/usr/lib/hadoop
export PIG_HOME=/root/pig-0.10.0
export SEQPIG_HOME=/root/seqpig 
\end{lstlisting}
%
\item To make life simpler also add the shell script directory to your path PATH:
%
\begin{lstlisting} 
export PATH=${PATH}:${SEQPIG_HOME}/bin
\end{lstlisting}
%
\item Finally, for convenience add the following line to your {\tt .bashrc}:
%
\begin{lstlisting} 
alias pig='${PIG_HOME}/bin/pig -Dpig.additional.jars=${SEQPIG_HOME}/lib/hadoop-bam-5.0.jar:${SEQPIG_HOME}/build/jar/SeqPig.jar:${SEQPIG_HOME}/lib/seal.jar:${SEQPIG_HOME}/lib/picard-1.76.jar:${SEQPIG_HOME}/lib/sam-1.76.jar -Dudf.import.list=fi.aalto.seqpig' 
\end{lstlisting}
%
Note that some of the example scripts below (e.g.,
Section~\ref{sect:read_clipping}) require functions from \emph{PiggyBank},
which is a collection of publicly available User-Defined Functions (UDF's)
that are distributed with Pig but need to be built separately.
For more details see
\url{https://cwiki.apache.org/confluence/display/PIG/PiggyBank}. If you would like to
make use of these you should also add {\tt piggybank.jar} to the library path, which
means the statement above becomes
\begin{lstlisting} 
alias pig='${PIG_HOME}/bin/pig -Dpig.additional.jars=${SEQPIG_HOME}/lib/hadoop-bam-5.0.jar:${SEQPIG_HOME}/build/jar/SeqPig.jar:${SEQPIG_HOME}/lib/seal.jar:${SEQPIG_HOME}/lib/picard-1.76.jar:${SEQPIG_HOME}/lib/sam-1.76.jar:${PIG_HOME}/contrib/piggybank/java/piggybank.jar -Dudf.import.list=fi.aalto.seqpig' 
\end{lstlisting}
\end{enumerate}

\subsection{Instructions for building SeqPig.jar}

\begin{enumerate}
\item Download hadoop-bam-5.0 from \url{https://sourceforge.net/projects/hadoop-bam/}.

\item Download and compile the latest biodoop/seal git master version from
 \url{http://biodoop-seal.sourceforge.net/}. Note that this requires
 setting {\tt HADOOP\_BAM} to the installation directory of hadoop-bam.

\item Inside the cloned git repository ({\tt \$SEQPIG\_HOME}), create a
{\tt lib/} subdirectory and copy the following jar files contained in the
hadoop-bam release to this location:
%
\begin{lstlisting} 
    seal.jar     hadoop-bam-5.0.jar     sam-1.76.jar     picard-1.76.jar
\end{lstlisting}
%
Note: the Picard and Sam jar files are contained in the hadoop-bam release
for convenience.

\item Run {\tt ant} to build {\tt SeqPig.jar}.
\end{enumerate}

\subsection{Usage}

\subsubsection{Pig grunt shell for interactive operations}
Assuming that all the environment variables have been set correctly, it suffices
to start the grunt shell via
%
\begin{lstlisting}
pig
\end{lstlisting}
%
\subsubsection{Starting scripts from the command line for non-interactive use}
Alternatively to using the Pig grunt shell (which can lead to delays due to
Hadoop queuing and execution delays), users can write scripts that are
then submitted to Pig/Hadoop for execution. This type of execution has
the advantage of being able to handle parameters, for example for input
and output files. See /scripts inside the seqpig directory and the
examples below.
