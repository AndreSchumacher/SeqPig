
\section{Examples}

\subsection{Operations on BAM files:}

All examples assume that an input BAM file is initially imported to HDFS via
\begin{lstlisting} 
prepareBamInput.sh input.bam
\end{lstlisting} 
and then loaded in the grunt shell via
\begin{lstlisting} 
grunt> A = load 'input.bam' using BamUDFLoader('yes');
\end{lstlisting} 
(the 'yes' chooses read attributes to be loaded; choose 'no' whenever these
are not required).

Once some operations have been performed, the resulting (modified) read
data can then be stored into a new BAM file via
\begin{lstlisting}
grunt> store A into 'output.bam' using BamUDFStorer('input.bam.asciiheader');
\end{lstlisting}
and can also be exported from HDFS to the local filesystem via
\begin{lstlisting}
prepareBamOutput.sh output.bam
\end{lstlisting}
(note: the Pig store operation requires a valid header for the BAM output file,
for example the header of the source file used to generate it, which is
generated automatically by the prepareBamInput.sh script used to import it)

Note that dumping the BAM data to the screen (similarly to samtools view)
can be done simply by
\begin{lstlisting}
grunt> dump A;
\end{lstlisting}
Another very useful Pig command is describe, which returns the schema that Pig
uses for a given data bag. Example:
\begin{lstlisting}
grunt> describe A;
\end{lstlisting}
which returns for BAM data
\begin{lstlisting}  
  A: {name: chararray,start: int,end: int,read: chararray,cigar: chararray,
   basequal: chararray,flags: int,insertsize: int,mapqual:int,matestart: int,
   materefindex: int,refindex: int,refname: chararray,attributes: map[]}
\end{lstlisting}
Note that all fields except the attributes are standard data types (strings
or integers). Specific attributes can be accessed via attributes\#'name', for
example
\begin{lstlisting} 
grunt> B = FOREACH A GENERATE name, attributes#'MD';
grunt> dump B;
\end{lstlisting}
will output all read names and their corresponding MD tag.
Other useful commands are LIMIT and SAMPLE, which can be used for example for obtaining
a subset of reads from a BAM/SAM file which can be useful for debugging.
\begin{lstlisting} 
grunt> B = LIMIT A 20;
\end{lstlisting}
will assign the first 20 records of A to B, while
\begin{lstlisting}
grunt> B = SAMPLE A 0.01;
\end{lstlisting}
will sample from A with sampling probability 0.01.

\subsubsection{Filtering out unmapped reads and PCR or optical duplicates}
\begin{lstlisting}
grunt> A = FILTER A BY (flags/4)%2==0 and (flags/1024)%2==0;
\end{lstlisting}

\subsubsection{Filtering out reads with low mapping quality}
\begin{lstlisting}
grunt> A = FILTER A BY mapqual > 19;
\end{lstlisting}
 \subsubsection{Filtering by regions (samtools syntax)}
\begin{lstlisting}
 grunt> DEFINE myFilter CoordinateFilter('input.bam.asciiheader','20:0-44350673');
 grunt> B = FILTER A BY myFilter(refindex,start,end);
\end{lstlisting}
Note that filtering by regions requires a valid ascii header for mapping
sequence names to sequence indices.

 \subsubsection{Sorting BAM files}
Sorting an input bam file by chromosome, reference start coordinate, strand
and readname (in this hierarchical order):
\begin{lstlisting}
grunt> A = FOREACH A GENERATE name, start, end, read, cigar, basequal, flags, insertsize,
mapqual, matestart, materefindex, refindex, refname, attributes, (flags/16)%2;
grunt> A = ORDER A BY refname, start, $14, name;
\end{lstlisting}
NOTE: this is roughly equivalent to executing from the command line:
\begin{lstlisting}
pig -param inputfile=input.bam -param outputfile=input_sorted.bam ${SEQPIG_HOME}/scripts/sort_bam.pig
\end{lstlisting}

 \subsubsection{Computing read coverage}
Computing read coverage over reference-coordinate bins of a fixed size,
for example:
\begin{lstlisting}
grunt> B = GROUP A BY start/200;
grunt> C = FOREACH B GENERATE group, COUNT(A);
grunt> dump C; 
\end{lstlisting}
will output the number of reads that lie in any non-overlapping bin of size 200 base pairs.

 \subsubsection{Computing base frequencies (counts) for each reference coordinate}
\begin{lstlisting}
grunt> A = FOREACH A GENERATE read, flags, refname, start, cigar, mapqual;
grunt> A = FILTER A BY (flags/4)%2==0;
grunt> RefPos = FOREACH A GENERATE ReadRefPositions(read, flags, refname, start, cigar, basequal), mapqual;
grunt> flatset = FOREACH RefPos GENERATE flatten($0), mapqual;
grunt> grouped = GROUP flatset BY ($0, $1, $2);
grunt> base_counts = FOREACH grouped GENERATE group.chr, group.pos, group.base, COUNT(flatset);
grunt> base_counts = ORDER base_counts BY chr,pos;
grunt> store base_counts into 'input.basecounts';
\end{lstlisting}
NOTE: this is roughly equivalent to executing from the command line:
\begin{lstlisting}
$ pig -param inputfile=input.bam -param outputfile=input.basecounts -param pparallel=1 ${SEQPIG_HOME}/scripts/basefreq.pig 
\end{lstlisting}

\subsubsection{Pileup}
Generating samtools compatible pileup (for a correctly sorted BAM file
with MD tags aligned to the same reference, should produce the same output as
samtools mpileup -A -f ref.fasta -B input.bam):
\begin{lstlisting}
grunt> A = load 'input.bam' using BamUDFLoader('yes');
grunt> B = FILTER A BY (flags/4)\%2==0 and (flags/1024)\%2==0;
grunt> C = FOREACH B GENERATE ReadPileup(read, flags, refname, start, cigar,
      basequal, attributes#'MD', mapqual), start, flags, name;
grunt> C = FILTER C BY $0 is not null;
grunt> D = FOREACH C GENERATE flatten($0), start, flags, name;
grunt> E = GROUP D BY (chr, pos);
grunt> F = FOREACH E { G = FOREACH D GENERATE refbase, pileup, qual, start,
      (flags/16)\%2, name; G = ORDER G BY start, $4, name; GENERATE group.chr,
      group.pos, PileupOutputFormatting(G, group.pos); }
grunt> F = ORDER F BY chr, pos;
grunt> G = FOREACH F GENERATE chr, pos, flatten($2);
grunt> store G into 'input.pileup' using PigStorage('\t');
\end{lstlisting}
NOTE: this is equivalent to executing from the command line:
\begin{lstlisting}
$ pig -param inputfile=input.bam -param outputfile=input.pileup -param pparallel=1
    ${SEQPIG_HOME}/scripts/pileup.pig
\end{lstlisting}
There are two optional parameters for pileup.pig: {\tt min\_map\_qual} and
{\tt min\_base\_qual} (both with default value 0) that filter out reads with
either insufficient map quality or base qualities. Their values can
be set the same way as the other parameters above.

\subsubsection{Collecting read-mapping-quality statistics}
In order to evalulate the output of an aligner, it may be useful to
consider the distribution of the mapping quality over the collection of
reads. Due to Pig's GROUP operator this is fairly easy.
\begin{lstlisting}
grunt> A = load 'input.bam' using BamUDFLoader('yes');
grunt> B = FILTER A BY (flags/4)\%2==0 and (flags/1024)\%2==0;
grunt> read_stats_data = FOREACH B GENERATE mapqual;
grunt> read_stats_grouped = GROUP read_stats_data BY mapqual;
grunt> read_stats = FOREACH read_stats_grouped GENERATE group, COUNT($1);
grunt> read_stats = ORDER read_stats BY group;
grunt> STORE read_stats into 'mapqual_dist.txt';
\end{lstlisting}
NOTE: this is equivalent to executing from the command line:
\begin{lstlisting}
$ pig -param inputfile=input.bam -param outputfile=mapqual_dist.txt ${SEQPIG_HOME}/scripts/read_stats.pig
\end{lstlisting}

\subsubsection{Collecting per-base statistics of reads}
Sometimes it may be useful to analyze a given set of reads for a bias
towards certain bases being called at certain positions inside the
read. The following simple script generates for each reference base and
each position inside a read the distribution of the number of read bases
that were called.
\begin{lstlisting}
grunt> A = load 'input.bam' using BamUDFLoader('yes');
grunt> B = FILTER A BY (flags/4)\%2==0 and (flags/1024)\%2==0;
grunt> C = FOREACH B GENERATE ReadSplit(name,start,read,cigar,basequal,flags,mapqual,refindex,refname,attributes#'MD');
grunt> D = FOREACH C GENERATE FLATTEN($0);
grunt> base_stats_data = FOREACH D GENERATE refbase, basepos, UPPER(readbase) AS readbase;
grunt> base_stats_grouped = GROUP base_stats_data BY (refbase, basepos, readbase);
grunt> base_stats_grouped_count = FOREACH base_stats_grouped GENERATE group.$0 AS refbase, group.$1 AS basepos, group.$2 as readbase, COUNT($1) AS bcount;
grunt> base_stats_grouped = GROUP base_stats_grouped_count by (refbase, basepos);
grunt> base_stats = FOREACH base_stats_grouped {
        TMP1 = FOREACH base_stats_grouped_count GENERATE readbase, bcount;
        TMP2 = ORDER TMP1 BY bcount desc;
        GENERATE group.$0, group.$1, TMP2;
   }
grunt> STORE base_stats into 'outputfile_readstats.txt';
\end{lstlisting}
A possible output has the form (for a BAM file with 50 reads):
\begin{lstlisting}
A       0       {(A,19),(G,2)}
A       1       {(A,10)}
A       2       {(A,18)}
A       3       {(A,16)}
A       4       {(A,14)}
A       5       {(A,15)}
A       6       {(A,16),(G,2)}
...
A       98      {(A,7)}
A       99      {(A,14)}
C       0       {(C,6)}
C       1       {(C,11)}
C       2       {(C,9)}
...
\end{lstlisting}
NOTE: this example script is equivalent to executing from the command line:
\begin{lstlisting}
$ pig -param inputfile=input.bam -param outputfile=outputfile_readstats.txt $SEQPIG_HOME/scripts/basequal_stats.pig
\end{lstlisting}

\subsubsection{Collecting per-base statistics of basequalities for reads}
Similarly as previously for the read bases themselves, we can also collect
frequencies for base-qualities depending on the position of the base inside
the reads. If these fall off too quickly for later positions, it may
indicate some quality issues with the run. The resulting script is actually
fairly similarly as the previous one with the difference of not grouping
over the reference bases.
\begin{lstlisting}
grunt> A = load 'input.bam' using BamUDFLoader('yes');
grunt> B = FILTER A BY (flags/4)\%2==0 and (flags/1024)\%2==0;
grunt> C = FOREACH B GENERATE ReadSplit(name,start,read,cigar,basequal,flags,mapqual,refindex,refname,attributes#'MD');
grunt> D = FOREACH C GENERATE FLATTEN($0);
grunt> base_stats_data = FOREACH D GENERATE basepos, basequal;
grunt> base_stats_grouped = GROUP base_stats_data BY (basepos, basequal);
grunt> base_stats_grouped_count = FOREACH base_stats_grouped GENERATE group.$0 as basepos, group.$1 AS basequal, COUNT($1) AS qcount;
grunt> base_stats_grouped = GROUP base_stats_grouped_count BY basepos;
grunt> base_stats = FOREACH base_stats_grouped {
        TMP1 = FOREACH base_stats_grouped_count GENERATE basequal, qcount;
        TMP2 = ORDER TMP1 BY basequal;
        GENERATE group, TMP2;
}
grunt> STORE base_stats into 'outputfile_basequalstats.txt';
\end{lstlisting}
A possible output has the form (for a BAM file with 50 reads):
\begin{lstlisting}
0       {(37,10),(42,1),(51,20),(52,1),(59,1),(61,1),(62,1),(67,2),(68,2),(70,2),(71,4),(72,3),(73,1),(75,2)}
1       {(53,1),(56,1),(61,1),(63,1),(64,1),(65,2),(67,4),(68,3),(69,2),(70,7),(71,3),(72,3),(73,1),(74,4),(75,2),(76,5),(77,6),(78,2),(80,1)}
2       {(45,1),(46,1),(51,2),(57,1),(61,1),(65,2),(66,3),(67,2),(69,3),(71,4),(72,2),(73,6),(74,7),(75,1),(76,8),(77,2),(78,3),(80,1)}
3       {(58,1),(59,1),(60,1),(61,1),(62,1),(64,1),(65,2),(67,2),(68,1),(69,5),(70,1),(71,3),(72,7),(73,2),(74,4),(75,6),(76,2),(77,4),(78,3),(79,1),(81,1)}
4       {(55,1),(60,1),(61,1),(62,1),(64,1),(66,1),(67,3),(68,2),(69,1),(70,7),(71,2),(72,1),(73,4),(74,2),(75,2),(76,2),(77,2),(78,3),(79,7),(80,4),(81,2)}
5       {(51,1),(52,2),(54,1),(58,2),(62,2),(63,1),(66,3),(68,4),(70,1),(71,1),(72,2),(73,3),(74,1),(75,8),(76,1),(77,5),(78,1),(79,6),(80,3),(81,3)}
...
\end{lstlisting}
NOTE: this example script is equivalent to executing from the command line:
\begin{lstlisting}
$ pig -param inputfile=input.bam -param outputfile=outputfile_basequalstats.txt' $SEQPIG_HOME/scripts/basequal_stats.pig
\end{lstlisting}

\subsubsection{Filtering reads by mappability threshold}
The script {\tt filter\_mappability.pig} filters reads in a given BAM file based on a given
mappability threshold. Both, input BAM and mappability file need to reside inside HDFS
\begin{lstlisting}
$ pig -param inputfile=/user/hadoop/input.bam -param outputfile=/user/hadoop/output.bam -param regionfile=/user/hadoop/mappability.100kbp.txt -param threshold=90 $SEQPIG_HOME/scripts/filter_mappability.pig
\end{lstlisting}
Note that since the script relies on distributing the bam file header and the mappability file via Hadoop's distributed cache,
it is not possible to run it in Pig's local mode.

\subsection{Processing Qseq and Fastq data}

SeqPig supports the import and export of non-aligned reads stored in Qseq and
Fastq data. Due to Pig's philosophy that all records correspond to tuples, which
form bags, reads can be processed in very much the same way independent
on for example whether they are stored in Qseq or Fastq.

\subsubsection{Converting Qseq to Fastq and vice versa}

The following two lines simply convert an input Qseq into Fastq.
\begin{lstlisting}
grunt> reads = load 'input.qseq' using QseqUDFLoader();
grunt> STORE reads INTO 'output.fastq' using FastqUDFStorer(); 
\end{lstlisting}
The other direction works analogously.

\subsubsection{Clipping bases and basequalities}

\label{sect:read_clipping}

Assuming there were some problems in certain cycles of the sequencer, it
may be useful to clip bases from reads. This example removes the last 3
bases and their qualities and stores the data under a new filename. Note
that here we rely on the SUBSTRING and LENGTH string functions, which is
part of the PiggyBank (see Section~\ref{sect:install_env}).

\begin{lstlisting}
grunt> DEFINE SUBSTRING org.apache.pig.piggybank.evaluation.string.SUBSTRING();
grunt> DEFINE LENGTH org.apache.pig.piggybank.evaluation.string.LENGTH();
grunt> reads = load 'input.qseq' using QseqUDFLoader();
grunt> B = FOREACH A GENERATE instrument, run_number, flow_cell_id, lane, tile, xpos, ypos, read, qc_passed, control_number, index_sequence, SUBSTRING(sequence, 0, LENGTH(sequence) - 3) AS sequence, SUBSTRING(quality, 0, LENGTH(quality) - 3) AS quality;
grunt> store B into 'output.qseq' using QseqUDFStorer();
\end{lstlisting}
NOTE: this example script is equivalent to executing from the command line:
\begin{lstlisting}
$ pig -param inputfile=input.qseq -param outputfile=output.qseq -param backclip=3 $SEQPIG_HOME/scripts/clip_reads.pig
\end{lstlisting}

\section{Further information}

\subsection{Compression}

For performance reasons it can be advisable to enable compression of
Hadoop map (and possible reduce) output, as well as temporary data generated
by Pig. The details depend on which compression codecs are used, but it can
be enabled by passing parameters along the lines of
\begin{lstlisting}
  -Djava.library.path=/opt/hadoopgpl/native/Linux-amd64-64
  -Dpig.tmpfilecompression=true -Dpig.tmpfilecompression.codec=lzo
  -Dmapred.output.compress=true
  -Dmapred.output.compression.codec=org.apache.hadoop.io.compress.GzipCodec
\end{lstlisting}

 to the pig command. Note that currently not all Hadoop compression codecs are
 supported by Pig.

For more examples see also the wiki of two past COST hackathons:\\
\url{http://seqahead.cs.tu-dortmund.de/meetings:fastqpigscripting}\\
\url{http://seqahead.cs.tu-dortmund.de/meetings:2012-05-hackathon:pileuptask}\\
\url{http://seqahead.cs.tu-dortmund.de/meetings:2012-05-hackathon:seqpig_life_savers_page}

